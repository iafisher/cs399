\documentclass{article}

\usepackage{amsmath}
\usepackage{listings}

\lstset{
	language=Lisp,
	numbers=left,
	showstringspaces=false,
}

\bibliographystyle{acm}

\title{CS399 Thesis Draft 1}
\author{Ian Fisher}
\date{25 February 2019}

\begin{document}
\maketitle

\section{Introduction}
This thesis reviews abstract interpretation as a technique for analyzing programs that use faceted execution, a programming language mechanism for enforcing privacy policies. It presents an overview of faceted execution and abstract interpretation, and analyzes in-depth a sample program that uses faceted execution.

\section{Background}
\subsection{Faceted execution}
Faceted execution is a programming-language mechanism that allows sensitive data to be enclosed in special data structures called facets---tuples of the form $\langle l\ ?\ v_H : v_L \rangle$ where $l$ is a label, $v_H$ is the high-confidentiality value, and $v_L$ is the low-confidentiality value. $v_H$ can only be accessed by observers that match the facet's label; other observers can only see $v_L$, which is typically some default value like $0$ or \texttt{null}.

Faceted execution allows privacy policies to be expressed separately from the implementation of the program, meaning that changes to the privacy policy can be made reliably with minimal modification of the application logic. Faceted execution is thus an implementation of policy-agnostic programming \cite{faceted}.

Full support of faceted execution requires changes to core language mechanisms like function application. Concretely, if the function \textit{square-root} were applied to the facet $\langle l \ ?\ 42 : 0 \rangle$, it would return the facet $\langle l \ ?\ \textit{square-root}(42) : \textit{square-root}(0) \rangle$, ensuring that the faceted value remains protected by the privacy policy, even if \textit{square-root} is totally oblivious to the policy.

Researchers have adopted different strategies to implement faceted execution. One strategy is to design a new programming language with faceted-execution primitives built-in. This is the strategy adopted by the Jeeves programming language \cite{jeeves}. Another strategy is to use syntactic macros to graft faceted execution on to an existing language, provided that the language's macro system is rich enough to support it. The \textsc{Racets} programming language adopts the latter strategy, by augmenting the Racket language with syntactic macros \cite{racets}.

The following subsections will present an overview of the specific mechanics of the \textsc{Racets} programming language, but the concepts are general enough to apply to other implementations of faceted execution.

\subsubsection{A simple example of faceted execution}
Policies governing access to data are declared with the \texttt{let-label} form in \textsc{Racets}:

% TODO: Double quotes do not display correctly.
\begin{lstlisting}
(define alice-policy
    (let-label l (lambda (x) (equal? x "Alice")) l))

(define bob-policy
    (let-label l (lambda (x) (equal? x "Bob")) l))
\end{lstlisting}

The two declarations in the source code above create two policies and bind them to the names \texttt{alice-policy} and \texttt{bob-policy}. The policies enforce that only entities identifying themselves as ``Alice'' or ``Bob'', respectively, may view the high-confidentiality value of any facet under the policies.

A faceted data value is created with the \texttt{fac} form:

\begin{lstlisting}
(define my-facet (fac alice-policy 42 0))
\end{lstlisting}

\texttt{my-facet} is defined with Alice's policy, the high-confidentiality value $42$, and the low-confidentiality value $0$.

The \texttt{obs} form is used to view the value of a facet:

\begin{lstlisting}
(obs alice-policy "Alice" my-facet)
\end{lstlisting}

The expression above will evaluate to $42$, as the argument \texttt{"Alice"} satisfies the facet's policy. By contrast, the expression below will evaluate to $0$ since \texttt{"Bob"} does not satisfy the facet's policy.

\begin{lstlisting}
(obs alice-policy "Bob" my-facet)
\end{lstlisting}

The policy passed to a facet must match the policy that the facet was created with. In the case that the facets do not match, \texttt{obs} functions as a no-op (so that the semantics are sound). Each of the two calls to \texttt{obs} below, for instance, will simply return \texttt{my-facet} unchanged.

\begin{lstlisting}
(obs bob-policy "Alice" my-facet)
(obs bob-policy "Bob" my-facet)
\end{lstlisting}

\subsubsection{An example of nested facets}
Faceted values may be nested for more fine-grained control over the views of the data that are available. Alice may define a nested facet as follows:

\begin{lstlisting}
(define location-facet
  (fac alice-policy
    "370 Lancaster Ave, Haverford PA"
    (fac bob-policy
      "Haverford, PA"
      "Pennsylvania")))
\end{lstlisting}

Alice is able to view the full street address of her location. Bob (or anyone else satisfying Bob's policy) may see her town, and anyone else may only see her state.

Alice observes her nested facet in the usual way:

\begin{lstlisting}
(obs alice-policy "Alice" location-facet)
\end{lstlisting}

Bob must make two calls to \texttt{obs} to fully resolve the facet's value:

\begin{lstlisting}
(obs alice-policy "Bob" (obs bob-policy "Bob" location-facet))
\end{lstlisting}

As before, Bob must ensure that the policy he passes to each \texttt{obs} call matches the policy of the facet. In this case, the outer facet uses Alice's policy and the inner facet uses Bob's policy, so the calls to \texttt{obs} must be organized likewise.

\subsubsection{An example of faceted lists}
Using faceted values with data structures involves additional complications. Consider the following example:

\begin{lstlisting}
(define grades (list))
(set! grades (cons (fac alice-policy 84 0) grades))
\end{lstlisting}

The programmer likely intended for \texttt{grades} to be of the form \texttt{(list (fac alice-policy 84 0))}, i.e. a regular list containing a single facet. However, the rules of faceted execution with regard to function application apply to \texttt{cons} just like any other function: functions applied to faceted values yield faceted return values. The real value of \texttt{grades} after the \texttt{set!} operation is

\begin{lstlisting}
(fac alice-policy (list 84) (list 0))
\end{lstlisting}

Imagine another grade was added to the list, like so:

\begin{lstlisting}
(set! grades (cons (fac bob-policy 73 0) grades))
\end{lstlisting}

Then the value of the list would be

\begin{lstlisting}
(fac bob-policy
  (fac alice-policy (list 73 84) (list 73 0))
  (fac alice-policy (list 0 84) (list 0 0)))
\end{lstlisting}

encompassing four different possibilities: satisfying both Alice and Bob's policies, satisfying one or the other, or satisfying neither.

The following procedure observes the entire list of grades into a regular Racket list of integers:

\begin{lstlisting}
(define (reveal-grades-rec grade-list policy-list arg)
  (if (empty? policy-list)
      grade-list
      (obs
        (car policy-list)
        arg
        (reveal-grades-rec
          grade-list
          (cdr policy-list)
          arg))))])
\end{lstlisting}

It takes in a list of grades, a list of policies which should correspond index-by-index to the list of grades (remember that \texttt{obs} requires the policy to match the facet's policy), and an argument to pass to the policy predicates.

\subsubsection{Efficiency of faceted execution}
A practical problem of implementing faceted execution is that it can have high runtime cost. As the example with \textit{square-root} above indicates, functions applied to facets must be evaluated twice, once with the high-confidentiality value and once with the low-confidentiality value. The double evaluation imposes a significant runtime cost on faceted execution (as well as complicating the use of functions with side-effects). This cost can be mitigated to some extent by static analysis with an abstract interpreter. If the static analyzer can prove that a facet passed to a function only ever evaluates to its high-confidentiality value, then the evaluation of the function with the low-confidentiality value could be skipped, and significant performance gains could be realized.

\subsection{Abstract interpretation}
One technique for static analysis is abstract interpretation, wherein the computation of a program is modelled with abstract objects (e.g., the abstract ``negative number'' instead of the concrete $-10$) so that real properties of the program can be reasoned about \cite{ai-original}. Abstract interpreters can be derived mechanically from abstract machines, provided that the formalism for the abstract machine is expressed in an amenable manner \cite{aam}. An abstract interpreter of a \textsc{Racets}-like language is under development for this purpose \cite{abstract-inter}.

\subsection{Type theory}
As the example \textsc{Racets} programs presented above indicated, programmers must be careful to apply the correct policies when they observe faceted values. As the \textsc{Racets} language is dynamically typed, errors in the use of faceted execution are not caught until runtime. A static type system for faceted values would be able to more reliably catch programmer errors.

The definition of a type system is ``a tractable syntactic method for proving the absence of certain program behaviors by classifying phrases according to the kinds of values they compute'' \cite{types}. The program behaviors that a type system are meant to prevent are generally what are called type errors: operations on values for which the operation is not defined.

Naturally, some undesirable program behaviors cannot be detected by a static type system (e.g., looping infinitely is not detectable due to the halting problem), and type systems will sometimes refuse to accept constructs which are in fact safe.

\section{Research plan}
My research will proceed in two directions: abstract interpretation and type theory. On the abstract interpretation side of things, it will involve writing several substantial example programs that use faceted execution, for example a grade server in which professors, teacher's assistants, and students all have different levels of access to the grade data, protected by facets. Once I have written these programs in \textsc{Racets}, I will translate them into the language of the abstract interpreter, and assess what conclusions it is able to draw about them, and to what level of precision. For example, if a certain function is sometimes applied to faceted values and sometimes applied to regular values, the abstract interpreter may or may not be able to identify the call sites that require special handling of the faceted execution, and those that can be run normally. In general, the more fine-grained the conclusions the abstract interpreter can draw, the greater the improvement of performance time.

My research will aid in understanding the efficacy of static analysis as applied to faceted execution, and thus contribute to improving the programming-language resources that software developers have available to correctly and reliably enforce privacy policies.

\bibliography{thesis}

\end{document}
